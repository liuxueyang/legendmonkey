\documentclass[a4paper,11pt]{article}
\usepackage[T1]{fontenc}
\usepackage[utf8]{inputenc}
\usepackage{lmodern}

\title{IRC LISP}
\author{liuxueyang}

\begin{document}

\maketitle
\tableofcontents

\begin{abstract}
\end{abstract}

\section{}


<dto> hi Kooda and welcome to the channel.
<dto> im upgrading linuxmint so i might not get a chance to try your lua game before i pass out in exhaustion
<dto> how are you doing?
<dto> are you interested in Common Lisp? or Scheme? clojure? whatevs
<dto> it's all a rich tapestry
<dto> it's a dialect-agnostic scene
<Kooda> I’m a Schemer :Þ
<Kooda> I use (and try to contribute) to the CHICKEN implementation.
<dto> hey cool. i bet you would enjoy talking to "davexunit" here when he is around. he does a scheme game lib
<dto> hey cool :)
<dto> i think davexunit uses Guile
<Kooda> Awesome! I just rolled my own thing for my first game
<dto> he works on a project called Guix which is a scheme-based linux distro
<dto> scheme package manager, etc
<Kooda> Yeah, I know about Guix, it seems awesome
<dto> kool Kooda what did you roll?
<dto> something in scheme?
<Kooda> Yep
<Kooda> Using bindings to SDL2
<Kooda> It’s just a very small and simple game though, nothing fancy. ^^
<dto> sweet. is it possible to distribute precompiled binaries with Chicken?
<Kooda> Yep, indeed!
<Firedancer> Doesn't it compile to C?
<dto> small and simple are the best games often tho
<dto> i have no idea
* sjl has quit (Ping timeout: 240 seconds)
<dto> hi Firedancer
<Kooda> I was just making a windows binary when you contacted me :è)
<Kooda> :)
<dto> neato.
<Kooda> Firedancer: yep, it does :)
<dto> i make both the Linux and the windows binaries for my games using SBCL running on linux. that is, I use linux sbcl to get a linux64 binary, and then Wine to run the windows port of SBCL, i.e. "SBCL.EXE" and it spits out a MYGAME.EXE that works on Real Windows
<dto> i bet you could do that too
<Firedancer> I was wondering would it work if I used ECL to translate to C and make binary that way
<dto> how would that work if it's in scheme?
<vydd> hi all
<dto> hey vydd .
<Kooda> With CHICKEN I made a cross compiler, I just have to invoke it to make the windows binary. :)
<vydd> hey dto
<dto> Kooda: cool :)
<dto> how does it work?
<Firedancer> No I mean writing Common Lisp program using ECL
* zaquest (~zaquest@5.128.210.30) has joined
<dto> Firedancer: i'm sorry. i misread your comment as coming from Kooda.
<Kooda> You have to build a CHICKEN aware of your target system, and you also need a C cross-compiler for your target system.
<Kooda> I used mingw-w64 as the C compiler.
<dto> thats quite fancy!!
<dto> :)
<dto> vydd, Firedancer please say welcome to a new visitor.
<Firedancer> for Kooda or zaquest?
<vydd> just reading through logs :)
<Firedancer> But I am just noob who has just started his journey on the Common Lisp landscape so I feel my welcoming doesn't mean anything :D
<vydd> hey Kooda, is your game online now?
<Kooda> vydd: no, but it should be very soon :)
<dto> Firedancer: wrong! noobs must socialize and share
<Kooda> I’m just making the final adjustments. ^^
<vydd> great, do post it here when you're done!
<Firedancer> I can sosialize, but I have nothing to share yet ^^'
<Kooda> vydd: ok! :D
<vydd> and stick around, I think you're going to love this channel
<dto> cool Kooda do you think it will run in Wine? or do you have a linux64 binary?
<dto> welcome to our savage, decadent underworld of lisp games
<Kooda> I will make linux and windows binaries :)
<dto> oh. we should have a jam after new years.
<Kooda> The windows binary works in wine :)
<dto> (or maybe people will be free enough for a 48hour jam around newyears?)
<dto> the question is , how to run it.
<dto> ok Kooda keep us posted :)
<vydd> newyears jam would be fun
<dto> if i fall asleep then ping me tomorrow.
<vydd> we just need to coordinate
<Kooda> Are you all in the US? :Þ
<dto> vydd: one simple solution would be to host it on itch.io, the drawback is that people would have to have accounts.
<Firedancer> Not me Kooda :D
<vydd> I'm not
<dto> another decent choice IMO is GameJolt.
<dto> i'm in USA here
<dto> Kooda there are people from all over the world.
<aries> i'm in China :D
<vydd> hi aries
<aries> hello
<Kooda> Cool, people from everywhere! :)
<dto> over 60 names
<dto> hello aries.
<aries> hi, dto.
<vydd> dto: host the jam on itch.io?
<aries> i'm just a beginner for common lisp :(
<dto> vydd: yeah they have a thing. http://itch.io/jams
<dto> also gamejolt has a similar feature.
<Firedancer> I'm from Finland
<dto> i have all my stuff on both, for what it's worth. it's free and the support / admins are friendly / easy to deal with
<aries> How did you learn lisp?
<dto> i've noticed people reporting slow download speeds from itch lately, tho
<Firedancer> nice to have someone else that is completely new here aries o/
<aries> How long do you learn before you can write games in lisp.
<dto> aries: i got into lisp because of learning GNU Emacs in 2003-2004 and subsequently learning Emacs Lisp and writing some little games
<dto> then i went into Common Lisp.
<Kooda> aries: you can learn the language *by* making games ;)
<vydd> dto: I just created an account on itch.io
<aries> yeah. that's coll.
<dto> vydd: ok cool :)
<dto> vydd: it's a nice site. i have http://dto.itch.io
<dto> vydd: also we could use their forums they've just opened
<Kooda> Yeah, itch.io is nice and without too much frills
<dto> there is a comment feature on the individual pages
<aries> Is there some simple games written in lisp that I can learn from?
<dto> you can turn off comments on your game if you prefer
<dto> aries: i have a small 2d game-engine library thing, with a small example documented game https://github.com/dto/plong/blob/master/plong.lisp
<dto> it's a silly pong game but illustrates the basics.
<vydd> dto: Right, looks simple, I like it. We can do the jam there. I just followed you.
<dto> aries: i can't necessarily recommend using my library over other things, it doesn't do 3d, and it is out of date (uses obsolete opengl api, old version of SDL 1.2)
<dto> aries: but if you want to learn by messing around with that Plong, i will help you out.
<dto> you can transfer your learning to other CL game libs made by peoople here, or to Scheme / Clojureetc, if you suddenly discover that my library sucks :)
<dto> aries: if you are more into scheme then you could talk to Davexunit when he is here.
<dto> vydd: ok we're on!
<dto> for the jam
<aries> dto: ok, thank you.
<dto> we can decide exact dates sometime
<aries> this weekend?
<vydd> Sure, in a week maybe? We could use doodle
<dto> aries: the users |3b| and axion and vydd and oGMo are just some of the people working on various libraries/utilities. there's a new pretty fully-fleshed out SDL2 binding for Common Lisp (it even has MIXER and IMAGE) , and a utility library for that, and |3b| has worked a lot on Shaders and 3d related stuff
<dto> what's doodle
<dto> well here's the thing, this weekend is in a few days, so that's rather short notice.
<vydd> dto: http://doodle.com
<vydd> yeah, I vote for 20dec-10jan range
<dto> however aries if you want to do a game this weekend then Do The Thing , don't wait for the jam :)
<Kooda> (heh, doodle is also a tiny game library for CHICKEN ;) )
<dto> or you and 1-4 other newpersons can have a microjam
<dto> and then try to develop your ideas further at the actual jam
<dto> vydd: hmmmmm doodle
<aries> dto: thank you so much. (This is my blog long long ago: liuxueyang.github.io, did not update for few months...)
<dto> i'm reading this , maybe they have a tool http://itch.io/docs/creators/game-jams
<dto> aries: looking now
<vydd> dto: we don't have to use that; it was useful in a business setting for me, so I thought maybe we could use it for this as well
<dto> vydd: i'll peep further, it looks nice actually
<Kooda> If you prefer open-source services, there is : https://framadate.org/
<dto> aries: nice page! (i can read the english parts)
<aries> :) I am Chinese..
<aries> just a beginner..
<dto> welcome to the channel aries :)
<dto> ok. i propose a non-voting, non-ranked contest. just everybody DoThings(TM) and share. in my opinion.
<Kooda> These are the best jams ;)
<aries> Thank you. I am looking forward to learning lisp and write little games. 
<dto> with special focus on helping out with bugs or learner questions
<vydd> incf
<dto> Kooda: :)
<dto> aries: welcome! :)
<vydd> dto: So, are you going to be the main organizer? :) framadate.org that Kooda suggested looks nice too, so you can use that if you prefer open source
<dto> basically, if you run into problems and can't finish it or get some feature working, and the deadline's up, then submit what you have, and we can discuss / postmortem things and try to help fix such bugs afterward,
<dto> Sure i'll do it.
<vydd> \o/
<dto> i think i should make vydd and one or two others admins on the itch jam file so that if i get hit by a bus, it still happens
<dto> cool it comes with its own message board. http://blog.itch.io/post/128643409844/introducing-jam-communities
<vydd> Sure, add regulars
<dto> i don't think there should be a requirement that people make binaries. instead it might make sense to allow distributing source tarballs
<dto> on itch.
<dto> ok gimme a minute :)
* vydd just realized he sees himself as a regular here, and it's been only, what, two months? :)
<dto> :)
<vydd> what license says "do whatever you want with this, but if you're doing a multi-million dollar business, add me to your about section"?
<Kooda> CC-By?
<Kooda> Not sure it applies well to source-code though.
<vydd> Hm. It needs a link to the license, and one has to "indicate if changes were made". I don't really care about that. Or maybe I do? Then again, that's probably good when propagating the code through libraries.
<Kooda> LGPL can apply quite well too
<dto> just thought of something aries and Kooda,
<dto> there are a bunch of games in Emacs Lisp including tetris
<dto> whose source you can read/etc
* newcup has quit (Ping timeout: 260 seconds)
<aries> dto: what?
<Kooda> I wrote a tetris in CHICKEN a few weeks ago ^^
<vydd> Hm. After reconsideration, I think I'm just going to stick with MIT. Thanks, Kooda
<aries> Kooda:sounds interesting.
<dto> aries: do you use Emacs?
* sjl (~sjl@clients-pool-1.nat.ru.is) has joined
<Kooda> vydd: yeah, I tend to think the simpler the better ^^'
<aries> I am a vimer for two years. But I began to use Emacs two months ago. 
<vydd> Kooda: do you know about https://dthompson.us/pages/software/sly.html ? that's davexunit's library dto talked about earlier
<dto> aries: go into a new emacs window and M-x tetris :)
<Kooda> Ah! He’s the sly guy! :D
<Kooda> I know about Sly but I never used it.
<aries> while, I played it before.
<Kooda> Though I did look at his code, like when I tried to implement a functionnal reactive programming library.
<aries> dto: this is my github account: https://github.com/liuxueyang?tab=repositories :)
<dto> sweet! i followed you
<aries> dto: i followed you too :)
<vydd> followed :)
<aries> followed too. :)
<aries> dto: skyw0r package is so large.
<dto> aries: yea
<dto> lots of art
<aries> 88.0M for linux. Network speed is 4kB/s
<dto> aries: http://gamejolt.com/games/skyw0r/58813   faster here
<dto> vydd: one slight complication is that apparently itch.io is really slow these days
<dto> on downloads
<dto> aries is the 3rd person to tell me it was slow
<dto> or 4th
<aries> dto: I will try it.
* newcup (newcup@peruna.fi) has joined
<aries> dto: I can not connect gamejolt.com; i am in China, maybe because of the GFW(Great Fire Wall)
<dto> aries: oh :(
<aries> so sad. :(
<Kooda> :/
<Kooda> dto: you can host your itch.io downloads elsewhere
<dto> Kooda: i'm considering it
<dto> still it would be nice for everyone in the jam to be able to use itch to host. although source tarballs should be small enough to still download quick ...
<Kooda> Did you contact the itch.io guys regarding your bandwidth problem?
<dto> nope but i will
<Kooda> They are nice, I’m sure they will see to it
<Kooda> And, yeah, for the game jam, I’m sure it will be fine
<dto> ok, sent a message.
<Kooda> Anyways, thanks for inviting me over here. :D
<dto> welcome sir :)
<Kooda> I’m trying to start skyw0r but it complains that it can’t find SDL mixer although it is installed :(
<Kooda> Nevermind, it wanted headers
<dto> Kooda: if it still complains about GFX, try choosing the continue option
<Kooda> It works now :x
<dto> hope you enjoy :) it's sort of a messy work in progress
<Kooda> Heh, no problem with that :D
* derrida has quit (Ping timeout: 260 seconds)
<Kooda> Do you use speech synthesis for the voice, or is it recorded?
* derrida (~derrida-f@mgsarch.com) has joined
* derrida has quit (Changing host)
* derrida (~derrida-f@unaffiliated/deleuze) has joined
<dto> i use lisp code which shells out to command-line voice synth to make everything into WAV files.
<Kooda> Heh :D
* yrk (~user@c-50-189-99-166.hsd1.nh.comcast.net) has joined
* yrk has quit (Changing host)
* yrk (~user@pdpc/supporter/student/yrk) has joined
<Kooda> Well, I’m dead. :D
* warweasle (warweasle@2600:3c03::f03c:91ff:fe26:d1db) has joined
<warweasle> Firedancer: Afternoon.
<Firedancer> Afternoon warweasle :)
<Firedancer> I could only be half a hour in the exercise class. ^^' I always forget how bad my constitution(?) is
<vydd> hi warweasle
<warweasle> vydd: Hey ya!
<vydd> what's up? :)
<warweasle> vydd: I'm beating my head against the wall trying to figure out how to implement my idea.
<warweasle> vydd: Damn near literally.
<vydd> Your 6 or 7 ideas, you mean? :)
<warweasle> vydd: Well, the mixing code and data thing. It only works when you have some pretty major restrictions.
<warweasle> vydd: The diff/patch works. That might be useful still.
<vydd> Have you looked into JIT?
<warweasle> vydd: Yes. My initial idea was to evaluate all the static parts of the code, which use the available variables and send that code to run on the server.
<vydd> warweasle: we're organizing a game jam http://itch.io/jam/january-2016-lisp-game-jam !!
* zaquest has quit (Quit: Leaving)
<warweasle> vydd: But that breaks down when that code pulls data from something which needw outside data from another buffer or from a constantly changing source.
<vydd> How do you know which parts are tatic?
<vydd> +s
<warweasle> vydd: I was going to make a metacircular evaluator.
<warweasle> vydd: See what had constant values and only eval/macroexpand them.
<dto> hi warweasle
<warweasle> dto: Hey dto!
<dto> vydd: i added a simple abstract gamey background to the page and
<dto> altered the link color
* zaquest (~zaquest@5.128.210.30) has joined
<vydd> warweasle: Yes, but I was thinking about side effects via special variables that we talked about. Also can you try doing that without client-server architecture?
<vydd> dto: can we add just a little tiny bit of color?
<warweasle> vydd: That's not hard. It becomes a lambda function.
<vydd> pixelated stuff is great, but I don't know about black
<warweasle> vydd: I have a dynamic let (dlet) which can set up the environment for me.
<vydd> warweasle: Looking at other environments/systems, which one would you say is the most similar to 3Dmacs?
<warweasle> vydd: Modern html browsers.
<warweasle> vydd: Modern = with javascript
<dto> vydd: hmm hangon
<dlowe> so, post 1995?
<dlowe> ;)
* davexunit (~user@fsf/emeritus/davexunit) has joined
<vydd> warweasle: maybe try implementing something more similar to a web browser, do your qix-css and qix-html, use lisp instead of javascript, then see how that works?
<warweasle> dlowe: Compared to medieval browsers, from 1000 years ago. You remember the Guttenberg browser?
<dlowe> It was im-press-ive

\end{document}
